\documentclass[12pt]{report}
\usepackage[latin1]{inputenc}
\usepackage{multicol}
\usepackage{graphicx}
\graphicspath{ {images/} }

\vspace{3cm}
       
\title{AI1100 ASSIGNMENT1} 
\author{AI21BTECH11016} 
\date{March 2022}     

\begin{document}

\maketitle

\section{Problem 4-c}

\begin{multicols}{2}

\begin{itemize}

\item Drawing a circle of radius $4 cm$. Taking a point P outside the circle at a distance of $7 cm$ from the centre of the circle and constructing a pair of tangents to the circle from that point using Python. \\

\end{itemize}

\end{multicols}{2}

\def\inputGnumericTable{}

\textbf{Solution:}
\text{The input parameters for this construction are available in table}
\eqref{table:circ-tangent}

\begin{table}[ht!]

    \documentclass[12pt%
			  %,landscape%
                    ]{report}
       \usepackage[latin1]{inputenc}
       \usepackage{fullpage}
       \usepackage{color}
       \usepackage{array}
       \usepackage{longtable}
       \usepackage{calc}
       \usepackage{multirow}
       \usepackage{hhline}
       \usepackage{ifthen}

	\begin{document}
	
\providecommand{\gnumericmathit}[1]{#1} 
\providecommand{\gnumericPB}[1]%
{\let\gnumericTemp=\\#1\let\\=\gnumericTemp\hspace{0pt}}
 \ifundefined{gnumericTableWidthDefined}
        \newlength{\gnumericTableWidth}
        \newlength{\gnumericTableWidthComplete}
        \newlength{\gnumericMultiRowLength}
        \global\def\gnumericTableWidthDefined{}
 \fi
\ifthenelse{\isundefined{\languageshorthands}}{}{\languageshorthands{english}}

\providecommand\gnumbox{\makebox[0pt]}
                       
\setlength{\bigstrutjot}{\jot}
\setlength{\extrarowheight}{\doublerulesep} 	

\setlongtables

\setlength\gnumericTableWidth{%
	53pt+%
	53pt+%
	82pt+%
	53pt+%
0pt}
\def\gumericNumCols{4}
\setlength\gnumericTableWidthComplete{\gnumericTableWidth+%
         \tabcolsep*\gumericNumCols*2+\arrayrulewidth*\gumericNumCols}
\ifthenelse{\lengthtest{\gnumericTableWidthComplete > \linewidth}}%
         {\def\gnumericScale{1*\ratio{\linewidth-%
                        \tabcolsep*\gumericNumCols*2-%
                        \arrayrulewidth*\gumericNumCols}%
{\gnumericTableWidth}}}%
{\def\gnumericScale{1}}

\ifthenelse{\isundefined{\gnumericColA}}{\newlength{\gnumericColA}}{}\settowidth{\gnumericColA}{\begin{tabular}{@{}p{53pt*\gnumericScale}@{}}x\end{tabular}}
\ifthenelse{\isundefined{\gnumericColB}}{\newlength{\gnumericColB}}{}\settowidth{\gnumericColB}{\begin{tabular}{@{}p{53pt*\gnumericScale}@{}}x\end{tabular}}
\ifthenelse{\isundefined{\gnumericColC}}{\newlength{\gnumericColC}}{}\settowidth{\gnumericColC}{\begin{tabular}{@{}p{82pt*\gnumericScale}@{}}x\end{tabular}}
\ifthenelse{\isundefined{\gnumericColD}}{\newlength{\gnumericColD}}{}\settowidth{\gnumericColD}{\begin{tabular}{@{}p{53pt*\gnumericScale}@{}}x\end{tabular}}

	\begin{center}
\begin{tabular}[c]{%
	b{\gnumericColA}%
	b{\gnumericColB}%
	b{\gnumericColC}%
	b{\gnumericColD}%
	}

\hhline{|-|-|-~}
	 \multicolumn{1}{|p{\gnumericColA}|}%
	{\gnumericPB{\centering}\gnumbox{\textbf{Symbol}}}
	&\multicolumn{1}{p{\gnumericColB}|}%
	{\gnumericPB{\centering}\gnumbox{\textbf{Value}}}
	&\multicolumn{1}{p{\gnumericColC}|}%
	{\gnumericPB{\centering}\gnumbox{\textbf{Description}}}
	&
\\
\hhline{|---|~}
	 \multicolumn{1}{|p{\gnumericColA}|}%
	{\gnumericPB{\centering}\gnumbox{$r$}}
	&\multicolumn{1}{p{\gnumericColB}|}%
	{\gnumericPB{\centering}\gnumbox{${3.5}$}}
	&\multicolumn{1}{p{\gnumericColC}|}%
	{\gnumericPB{\centering}\gnumbox{$Radius$}}
	&
\\
\hhline{|---|~}
	 \multicolumn{1}{|p{\gnumericColA}|}%
	{\gnumericPB{\centering}\gnumbox{$d$}}
	&\multicolumn{1}{p{\gnumericColB}|}%
	{\gnumericPB{\centering}\gnumbox{${7}$}}
	&\multicolumn{1}{p{\gnumericColC}|}%
	{\gnumericPB{\centering}\gnumbox{Distance of $\vec{P} $  from the origin}}
	&
\\
\hhline{|---|~}
	 \multicolumn{1}{|p{\gnumericColA}|}%
	{\gnumericPB{\centering}\gnumbox{$sin \theta$}}
	&\multicolumn{1}{p{\gnumericColB}|}%
	{\gnumericPB{\centering}\gnumbox{$\frac{r}{d}$}}
	&\multicolumn{1}{p{\gnumericColC}|}%
	{\gnumericPB{\centering}\gnumbox{Angle between the tangent  from $\vec{P}$ and $d$}}
	&
\\
\hhline{|---|~}
	 \multicolumn{1}{|p{\gnumericColA}|}%
	{\gnumericPB{\centering}\gnumbox{$\vec{P}$}}
	&\multicolumn{1}{p{\gnumericColB}|}%
	{\gnumericPB{\centering}\gnumbox{$\vec{0}$}}
	&\multicolumn{1}{p{\gnumericColC}|}%
	{\gnumericPB{\centering}\gnumbox{Origin}}
	&
\\
\hhline{|---|~}
	 \multicolumn{1}{|p{\gnumericColA}|}%
	{\gnumericPB{\centering}\gnumbox{$\vec{O}$}}
	&\multicolumn{1}{p{\gnumericColB}|}%
	{\gnumericPB{\centering}\gnumbox{$\myvec{d \\ 0}$}}
	&\multicolumn{1}{p{\gnumericColC}|}%
	{\gnumericPB{\centering}\gnumbox{Centre of the circle}}
	&
\\
\hhline{|---|~}
	 \multicolumn{1}{|p{\gnumericColA}|}%
	{\gnumericPB{\centering}\gnumbox{$\vec{Q}_i$}}
	&\multicolumn{1}{p{\gnumericColB}|}%
	{\gnumericPB{\centering}\gnumbox{$r\cot \theta \myvec{\cos \theta\\ \pm \sin \theta}$}}
	&\multicolumn{1}{p{\gnumericColC}|}%
	{\gnumericPB{\centering}\gnumbox{Points of Contact}}
	&
\\
\hhline{|-|-|-|~}
\end{tabular}
	\end{center}

\ifthenelse{\isundefined{\languageshorthands}}{}{\languageshorthands{\languagename}}
\gnumericTableEnd
\caption{}
  \label {table:circ-tangent}
	    
	
  \begin{figure}[h!]
	  \centering 
	  \includegraphics[width=\columnwidth]{imagePython}
	  \caption{}
	  \end{figure}
	  
\newpage
	  
\begin{multicols}{2}	  
	  	  	  
\begin{itemize}

\item Calculating the length of tangent $PQ2$, which is the third side of the triangle OQ2P, right-angled at Q2 using the Pythogorean theorem using C. \\

\end{itemize} 

\end{multicols}{2}
 
 \begin{figure}[h!]
	  \centering 
	  \includegraphics[width=\columnwidth]{imageC}
	  \caption{}
	  \end{figure}
	  
	  
	  	  	  
\textbf{Therefore, the length of tangent drawn from P on to the given circle is 5.744563.}	  



\end{document}


